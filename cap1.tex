
%=======================================================
\section[Pregunta Nro 1]{Pregunta Nro 1}
\begin{center}
%-------------------------Pregunta------------------------- 
{ \fboxsep 12pt
\fcolorbox {blue}{white}{
\begin{minipage}[t]{15,5cm}
\begin{flushleft}
\textbf{Enunciado:}
\end{flushleft}
Un taller tiene 5 empleados.Los salarios diarios en dolares de cada uno de ellos son:5,7,8,10,10.
%Enumeracion
\begin{description}
\item[a)] Determina la media y la varianza de la población.
\item[b)] Halle la distribución muestral de las medias para muestras de tamaño 2 escogidas (sin sustitución) de esta población.
\item[c)] Determine la media y la varianza de la distribución muestral de las medias de tamaño 2.
\item[d)] Compare la media de las medias muestras con la media de la población.También compare la dispersión de las medias de las muestras con la dispersión de la población.
\end{description}
\end{minipage}
} }
\end{center}

\begin{center}
%-------------------------Resolucion------------------------- 
{ \fboxsep 12pt
\fcolorbox {orange}{white}{
\begin{minipage}[t]{15,5cm}
\begin{flushleft}
\textbf{Resolución:}
\end{flushleft}
%\begin{enumerate}
%\item 
%\end{enumerate}
%Enumeracion
\begin{description}
\item[a)] Resolución a.
\begin{center}
\begin{tabular}{|c|c|c|c|c|c|} \hline
$X$ & 5 & 7 & 8 &10  & 10\\ \hline
$F(X)$ & 1/5 & 1/5 & 1/5 & 1/5 & 1/5  \\ \hline
\end{tabular}
\end{center}
$ \mu =\sum _{ i=1 }^{ N }{ { X }_{ i }f({ X }_{ i }) } =8\\ { \sigma  }^{ 2 }=\sum _{ i=1 }^{ N }{ { { X }_{ i } }^{ 2 }f({ X }_{ i }) } -{ \mu  }^{ 2 }=3.6 $
\item[b)] Resolución b.
\end{description}
\end{minipage}
} }
\end{center}

\section[Pregunta Nro 2]{Pregunta Nro 2}
\begin{center}
%-------------------------Pregunta------------------------- 
{ \fboxsep 12pt
\fcolorbox {blue}{white}{
\begin{minipage}[t]{15,5cm}
\begin{flushleft}
\textbf{Enunciado:}
\end{flushleft}
La utilidad por la venta de cierto artículo,en miles de soles, es una variable aleatoria con distribución normal. En el 5\% de las ventas la utilidad ha sido menos que 6.71, mientras que el 1\% ha sido mayor que 14.66. Si se realizan 16 operaciones de ventas, ¿cuál es la probabilidad de que el promedio de la utilidad por cada operación esté entre \$10.000 y \$11,000?.
%Enumeracion
%\begin{description}
%\item[a)] Enunciado a.
%\item[b)] Enunciado b.
%\end{description}
\end{minipage}
} }
\end{center}
\newpage

\section[Pregunta Nro 4]{Pregunta Nro 4}
\begin{center}
%-------------------------Pregunta------------------------- 
{ \fboxsep 12pt
\fcolorbox {blue}{white}{
\begin{minipage}[t]{15,5cm}
\begin{flushleft}
\textbf{Enunciado:}
\end{flushleft}
Una compañía agro-industrial ha logrado establecer el siguiente modelo de probabilidad discreta de los sueldos (X) en cientos de dolares de su personal:

\begin{center}
\begin{tabular}{|c|c|c|c|c|c|}
\hline 
x & 1 & 2 & 3 & 4 & 5 \\ 
\hline 
f(x)=P[X=x] & 0.1 & 0.2 & 0.4 & 0.2 & 0.1 \\ 
\hline 
\end{tabular} 
\end{center}
si de esta población de sueldos se toman 30 sueldos al azar:
%Enumeracion
\begin{description}
\item[a)] Halle la media y la varianza de la media muestral.
\item[b)] Calcule la probabilidad de que la media muestral esté entre 260 y 330 dólares.

\end{description}
\end{minipage}
} }
\end{center}
\newpage
\section[Pregunta Nro 5]{Pregunta Nro 5}
\begin{center}
%-------------------------Pregunta------------------------- 
{ \fboxsep 12pt
\fcolorbox {blue}{white}{
\begin{minipage}[t]{15,5cm}
\begin{flushleft}
\textbf{Enunciado:}
\end{flushleft}
La demanda diaria de un producto puede ser: 1, 2, 3, 4 con probabilidades respectivas: 0.3, 0.3, 0.2, 0.1, 0.1.
la utilidad por cada operación este entre 10.000 y 11,000?
%Enumeracion
\begin{description}
\item[a)] Describa el modelo de probabilidad de la demanda promedio de 36 días..
\item[b)] ¿Qué probabilidad hay de que la demanda promedio de 36 días este entre 1 y 2?.
\end{description}
\end{minipage}
} }
\end{center}

\begin{center}
%-------------------------Resolucion------------------------- 
{ \fboxsep 12pt
\fcolorbox {orange}{white}{
\begin{minipage}[t]{15,5cm}
\begin{flushleft}
\textbf{Resolución:}
\end{flushleft}

%Enumeracion
\begin{description}
\item[a)] Resolución a.
$ n=36\\ \mu _{ x }=E(x)=\Sigma xp(x)=0(0.3)+1(0.3)+2(0.2)+3(0.1)+4(0.1)\\ \mu _{ x }=1.4\\ E(x^{ 2 })=\sigma x^{ 2 }p(x)=02(0.3)+12(0.3)+22(0.2)+32(0.1)+42(0.1)\\ E(x^{ 2 })=3.6\\ VAR(x)=E(x^{ 2 })-\mu ^{ 2 }\\ \delta _{ x }^{ 2 }=3.6-1.4^{ 2 }\\ \delta _{ x }=0.045\\ N(1.4,\frac { 1.64 }{ 36 } ) $
\item[b)] Resolución b.
$ P(1\leq x\geq 2)=\phi (\frac { 2-1.4 }{ 0.21343 } )-\phi (\frac { 1-1.4 }{ 0.21343 } )\\ =\phi (2.81)-\phi (-1.87)\\ =\phi (2.81)-[1-\phi (1.87)]\\ =0.9668 $
\end{description}
\end{minipage}
} }
\end{center}

\newpage
\section[Pregunta Nro 6]{Pregunta Nro 6}
\begin{center}
%-------------------------Pregunta------------------------- 
{ \fboxsep 12pt
\fcolorbox {blue}{white}{
\begin{minipage}[t]{15,5cm}
\begin{flushleft}
\textbf{Enunciado:}
\end{flushleft}
Una empresa comercializadora de café sabe que el consumo mensual(en Kgr) de café por casa esta normalmente distribuida con una media desconocida u y una desviación estándar de 0.30. Si se toma una muestra aleatoria de 36 casas y se registra su consumo de café durante un mes,¿ cuál es la probabilidad de que la media de la muestra esté entre los valores u-0.1 y u+0.1?.
\end{minipage}
} }
\end{center}

\begin{center}
%-------------------------Resolucion------------------------- 
{ \fboxsep 12pt
\fcolorbox {orange}{white}{
\begin{minipage}[t]{15,5cm}
\begin{flushleft}
\textbf{Resolución:}
\end{flushleft}
$ \delta _{ x }=0.3,n=36\\ media:\mu \\ P(u-0.1\leq x\leq u+0.1)=\phi (\frac { u+0.1-u }{ \frac { 0.3 }{ 6 }  } )-\phi (\frac { u-0.1-u }{ \frac { 0.3 }{ 6 }  } )\\ =\phi (2)-\phi (-2)\\ =\phi (2)-[1-\phi (2)]\\ =0.9772-[1-0.9772]\\ =0.9544 $
\end{minipage}
} }
\end{center}

\section[Pregunta Nro 11]{Pregunta Nro 11}
\begin{center}
%-------------------------Pregunta------------------------- 
{ \fboxsep 12pt
\fcolorbox {blue}{white}{
\begin{minipage}[t]{15,5cm}
\begin{flushleft}
\textbf{Enunciado:}
\end{flushleft}
La utilidad en miles de soles por la venta de cierto articulo, es una variable
aleatoria con distribución normal. Se estima que en el \%5 de las ventas las utilidades
serian   al   menos de 67.1, mientras que el \%1 de las ventas serian mayores que 14.66.Si se
se realizan 16 operaciones de ventas, ¿cual es la probabilidad de que el promedio de
la utilidad por cada operación este entre 10.000 y 11,000?
%Enumeracion
%\begin{description}
%\item[a)] Enunciado a.
%\item[b)] Enunciado b.
%\end{description}
\end{minipage}
} }
\end{center}

\begin{center}
%-------------------------Resolucion------------------------- 
{ \fboxsep 12pt
\fcolorbox {orange}{white}{
\begin{minipage}[t]{15,5cm}
\begin{flushleft}
\textbf{Resolución:}
\end{flushleft}
$X:\quad "Utilidad\quad en\quad miles\quad de\quad soles"\\ x\rightarrow N(u,{ \delta  }_{ x }^{ 2 })\quad n=16\\ a.\quad P(x<6.71)=0.05\\ \qquad \Phi (\frac { 6.71-u }{ { \delta  }_{ x } } )=0.05\\ \qquad \frac { 6.71-u }{ { \delta  }_{ x } } =-1.645\\ \qquad \frac { u-6.71 }{ 1.645 } ={ \delta  }_{ x }....................1\\ b.\quad P(x>6.71)=0.01\\ \qquad 1-\Phi (\frac { 14.66-u }{ { \delta  }_{ x } } )=0.01\\ \qquad \frac { 14.66-u }{ { \delta  }_{ x } } =2.33\\ \qquad \frac { 14.66-u }{ 2.33 } ={ \delta  }_{ x }....................2\\ IGUALANDO\quad 1\quad Y\quad 2\\ u=10\\ { \delta  }_{ x }=2\\ P(10\le \quad x\quad \le 11)=?\\ \Phi (\frac { 11-10 }{ { 2/4 }_{ x } } )-\Phi (\frac { 10-10 }{ 2/4 } )\\ \Phi (2)-\Phi (0)\\ 0.9972-0.5\\ 0.4772\\ \therefore \quad P(10\le \quad x\quad \le 11)=0.4772 $
%Enumeracion
%\begin{description}
%\item[a)] Resolución a.
%\item[b)] Resolución b.
%\end{description}
\end{minipage}
} }
\end{center}

\newpage
\section[Pregunta Nro 12]{Pregunta Nro 12}
\begin{center}
%-------------------------Pregunta------------------------- 
{ \fboxsep 12pt
\fcolorbox {blue}{white}{
\begin{minipage}[t]{15,5cm}
\begin{flushleft}
\textbf{Enunciado:}
\end{flushleft}
La vida util de cierta marca de llantas radiales es una variable aleatoria X cuya distribución es normal con (i=38,000 Km. y c=3,000 Km.
%Enumeracion
\begin{description}
\item[a)] Si la utilidad Y (en dolares) que produce cada llanta esta dada por la relación:
Y=0.2X+100, ¿cual es la probabilidad de que la utilidad sea mayor que 8.900?.
\item[b)] Determinar el numero de tales llantas que debe adquirir una empresa de transporte para conseguir una utilidad promedio de almenos 7541 con probabilidad 0.996.
\end{description}
\end{minipage}
} }
\end{center}

\begin{center}
%-------------------------Resolucion------------------------- 
{ \fboxsep 12pt
\fcolorbox {orange}{white}{
\begin{minipage}[t]{15,5cm}
\begin{flushleft}
\textbf{Resolución:}
\end{flushleft}
$ x\quad \rightarrow N(3800,3000)\\ a)\quad \quad utilidad\quad en\quad dolares=y\\ \qquad y=0.2x+100\\ \qquad E(y)=0.2E(x)+100\\ \qquad E(y)=0.2(3800)+100\\ \qquad E(y)=7700\\ \qquad var(y)={ 0.2 }^{ 2 }var(x)\\ \qquad { \delta  }_{ y }=0.2{ \delta  }_{ x }=0.2(3000)=600\\ \qquad P(y>8900)=1-\Phi (\frac { 8900-7700 }{ 600 } )\\ \qquad \qquad \qquad \qquad =1-\Phi (2)\\ \qquad \qquad \qquad \qquad =1-0.9772\\ \qquad \qquad \qquad \qquad =0.0228\\ b)\quad P(y>7541)\\ \qquad E('y)=0.2E('x)+100\\ \qquad E('y)=0.2(3800)+100\\ \qquad var(y)=\frac { { \delta  }_{ x }^{ 2 } }{ n } \\ \qquad { \delta  }_{ y }=\frac { 600 }{ \sqrt { n }  } \\ \qquad P('y>7541)=1-\Phi (\frac { 7541-7700 }{ 600/\sqrt { n }  } )\\ \qquad 0.996=1-\Phi (\frac { 7541-7700 }{ 600/\sqrt { n }  } )\\ \qquad \Phi (2.65)=\Phi (\frac { 7541-7700 }{ 600/\sqrt { n }  } )\\ \qquad 2.65=\frac { 7541-7700 }{ 600/\sqrt { n }  } \\ \qquad n=100. $
%Enumeracion
%\begin{description}
%\item[a)] Resolución a.
%\item[b)] Resolución b.
%\end{description}
\end{minipage}
} }
\end{center}
\newpage
\section[Pregunta Nro 14]{Pregunta Nro 14}

\begin{center}
%-------------------------Pregunta------------------------- 
{ \fboxsep 12pt
\fcolorbox {blue}{white}{
\begin{minipage}[t]{15,5cm}
\begin{flushleft}
\textbf{Enunciado:}
\end{flushleft}
La utilidad por la venta de cierto artículo,en miles de soles, es una variable aleatoria con distribución normal. En el 5\% de las ventas la utilidad ha sido menos que 6.71, mientras que el 1\% ha sido mayor que 14.66. Si se realizan 16 operaciones de ventas, ¿cuál es la probabilidad de que el promedio de la utilidad por cada operación esté entre \$10.000 y \$11,000?.
%Enumeracion
%\begin{description}
%\item[a)] Enunciado a.
%\item[b)] Enunciado b.
%\end{description}
\end{minipage}
} }
\end{center}

\begin{center}
%-------------------------Resolucion------------------------- 
{ \fboxsep 12pt
\fcolorbox {orange}{white}{
\begin{minipage}[t]{15,5cm}
\begin{flushleft}
\textbf{Resolución:}
\end{flushleft}
$ X:\quad Utilidad\quad de\quad ventas\\ \bar { X } \rightarrow N(\mu ,{ \sigma  }^{ 2 }),\quad n=16 $
\begin{enumerate}
\item $ P(X<6.71)=0.05\\ P(\frac { X-\mu  }{ \sigma  } <\frac { 6.71-\mu  }{ \sigma  } )=0.05\\ \Phi (\frac { 6.71-\mu  }{ \sigma  } )=0.05\\ \frac { 6.71-\mu  }{ \sigma  } =-1.65\\ 6.71-\mu =-1.65\sigma \quad ----------(1) $
\item $ P(X>14.66)=0.01\\ P(\frac { X-\mu  }{ \sigma  } <\frac { 14.66-\mu  }{ \sigma  } )=0.01\\ 1-\Phi (\frac { 14.66-\mu  }{ \sigma  } )=0.01\\ \frac { 14.66-\mu  }{ \sigma  } =2.33\\ 14.66-\mu =2.33\sigma \quad ----------(2) $
\end{enumerate}
En (1) y (2)

$ 6.71-\mu =-1.65\sigma \\ 14.66-\mu =2.33\sigma \\ \mu =10\quad ,\quad \sigma =2 $

Entonces:

$ P(10<X<11)=?\\ P(\frac { 10-\mu  }{ \sigma  } \sqrt { 16 } <\frac { X-\mu  }{ \sigma  } \sqrt { 16 } <\frac { 11-\mu  }{ \sigma  } \sqrt { 16 } )=P(0<X<2)=P(2)-P(0)=0.9772-0.5\\ P(10<X<11)=0.4772 $
%Enumeracion
%\begin{description}
%\item[a)] Resolución a.
%\item[b)] Resolución b.
%\end{description}
\end{minipage}
} }
\end{center}
\newpage
\section[Pregunta Nro 15]{Pregunta Nro 15}

\begin{center}
%-------------------------Pregunta------------------------- 
{ \fboxsep 12pt
\fcolorbox {blue}{white}{
\begin{minipage}[t]{15,5cm}
\begin{flushleft}
\textbf{Enunciado:}
\end{flushleft}
La vida útil en meses de una batería es una variable aleatoria de X con distribución exponencial de parámetro $ \beta $ tal que, $P[X>5/X>2]=e^{ -0.6 }$

%Enumeracion
\begin{description}
\item[a)] Halle el valor de $\beta$ y la función de densidad.
\item[b)] ¿Cuantas baterías serán necesarias para que duren al menos 20 años con probabilidad 0.0228?.
\end{description}
\end{minipage}
} }
\end{center}

\begin{center}
%-------------------------Resolucion------------------------- 
{ \fboxsep 12pt
\fcolorbox {orange}{white}{
\begin{minipage}[t]{15,5cm}
\begin{flushleft}
\textbf{Resolución:}
\end{flushleft}
%\begin{enumerate}
%\item 
%\end{enumerate}
%Enumeracion
\begin{description}
\item[a)] Resolución a.
Función de densidad:\\
$ f(x)=\beta { e }^{ -\beta X },X\ge 0 $\\
Donde:\\
$ F(X)=P[X\le x]=\int _{ 0 }^{ x }{ \beta { e }^{ -\beta X }dt } =1-{ e }^{ -\beta x }\\ F(X)=P[X>x]=1-P[X\le x]={ e }^{ -\beta x } $\\
$P[X>5/X>2]=\dfrac{P(X>5)}{P(X>2)}=e^{ -0.6 }$\\
$ \frac { P(x>5) }{ P(x>2) } =\frac { { e }^{ -\beta 5 } }{ { e }^{ -\beta 2 } } ={ e }^{ -0.6 }\\ \therefore \quad -\beta 5+\beta 2=-0.6,\quad \beta =0.2 $
\item[b)] Resolución b.
\end{description}
\end{minipage}
} }
\end{center}
\newpage
\section[Pregunta Nro 16]{Pregunta Nro 16}
\begin{center}
%-------------------------Pregunta------------------------- 
{ \fboxsep 12pt
\fcolorbox {blue}{white}{
\begin{minipage}[t]{15,5cm}
\begin{flushleft}
\textbf{Enunciado:}
\end{flushleft}
En cierta población de matrimonios el peso en kilogramos de esposos y esposas se distribuye normalmente N(80,100) y N(64,69) respectivamente y son independientes. Si se eligen 25 matrimonios al azar de esa población calcular la probabilidad de que la media de los pesos sea a lo más 137 Kg.
%Enumeracion
\end{minipage}
} }
\end{center}

\begin{center}
%-------------------------Resolucion------------------------- 
{ \fboxsep 12pt
\fcolorbox {orange}{white}{
\begin{minipage}[t]{15,5cm}
\begin{flushleft}
\textbf{Resolución:}
\end{flushleft}
%\begin{enumerate}
%\item 
%\end{enumerate}
%Enumeracion
\begin{description}
\item[a)] Resolución a.
Función de densidad:\\
$ f(x)=\beta { e }^{ -\beta X },X\ge 0 $\\
Donde:\\
$ F(X)=P[X\le x]=\int _{ 0 }^{ x }{ \beta { e }^{ -\beta X }dt } =1-{ e }^{ -\beta x }\\ F(X)=P[X>x]=1-P[X\le x]={ e }^{ -\beta x } $\\
$P[X>5/X>2]=\dfrac{P(X>5)}{P(X>2)}=e^{ -0.6 }$\\
$ \frac { P(x>5) }{ P(x>2) } =\frac { { e }^{ -\beta 5 } }{ { e }^{ -\beta 2 } } ={ e }^{ -0.6 }\\ \therefore \quad -\beta 5+\beta 2=-0.6,\quad \beta =0.2 $
\item[b)] Resolución b.
\end{description}
\end{minipage}
} }
\end{center}

\newpage
\section[Pregunta Nro 16]{Pregunta Nro 16}
\begin{center}
%-------------------------Pregunta------------------------- 
{ \fboxsep 12pt
\fcolorbox {blue}{white}{
\begin{minipage}[t]{15,5cm}
\begin{flushleft}
\textbf{Enunciado:}
\end{flushleft}
Una empresa comercializa fardos de algodón cuyo peso X se distribuye normalmente con una media de 250 Kg y una desviación estándar de 4 Kg el costo por fardo es dado por Y = aX + 52. Hallar el valor de a si se quiere que la media de los costos de 4 fardos sea mayor que 3,100 con probabilidad 0.0228.
%Enumeracion
\end{minipage}
} }
\end{center}

\begin{center}
%-------------------------Resolucion------------------------- 
{ \fboxsep 12pt
\fcolorbox {orange}{white}{
\begin{minipage}[t]{15,5cm}
\begin{flushleft}
\textbf{Resolución:}
\end{flushleft}
%\begin{enumerate}
%\item 
%\end{enumerate}
%Enumeracion
\begin{description}
\item[a)] Resolución a.
Función de densidad:\\
$ f(x)=\beta { e }^{ -\beta X },X\ge 0 $\\
Donde:\\
$ F(X)=P[X\le x]=\int _{ 0 }^{ x }{ \beta { e }^{ -\beta X }dt } =1-{ e }^{ -\beta x }\\ F(X)=P[X>x]=1-P[X\le x]={ e }^{ -\beta x } $\\
$P[X>5/X>2]=\dfrac{P(X>5)}{P(X>2)}=e^{ -0.6 }$\\
$ \frac { P(x>5) }{ P(x>2) } =\frac { { e }^{ -\beta 5 } }{ { e }^{ -\beta 2 } } ={ e }^{ -0.6 }\\ \therefore \quad -\beta 5+\beta 2=-0.6,\quad \beta =0.2 $
\item[b)] Resolución b.
\end{description}
\end{minipage}
} }
\end{center}